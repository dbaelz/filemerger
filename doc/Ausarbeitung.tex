\documentclass[a4paper,titlepage,12pt]{scrartcl}

% utf-8
\usepackage{polyglossia}
\setdefaultlanguage[babelshorthands]{ngerman}
\usepackage{fontspec}

% german names
\usepackage{ngerman}

% colored links
\usepackage{color}
\usepackage[colorlinks]{hyperref}
% custom colors
\definecolor{grey}{rgb}{0.2,0.2,0.2}
\definecolor{orange}{rgb}{1,0.3,0}
\definecolor{turqoise}{rgb}{0,0.7,0.5}

% code listings
\usepackage{listings}
\lstset{%
	basicstyle={\ttfamily \small},
	breaklines=true,
	commentstyle=\color{grey},
	keywordstyle=\color{orange},
	language=C,
	numbers=left,
	showspaces=false,
	stringstyle=\color{turqoise},
	xleftmargin=20pt
}

% graphics
\usepackage{graphicx}
\graphicspath{img/}

% fancy headers and footers
\usepackage{fancyhdr}
\pagestyle{fancy}
% clear style
\fancyhead{}
\fancyfoot{}
% new style
\renewcommand{\headrulewidth}{0.5pt}
\renewcommand{\footrulewidth}{0.5pt}
\fancyhead[LE,RO]{\rightmark}
\fancyhead[LO,RE]{\leftmark}
\fancyfoot[LE,RO]{\thepage}
\fancyfoot[LO,RE]{Qualitätssicherung Wintersemester 2012/2013}
\fancypagestyle{plain}{%
	\fancyhf{}
	\renewcommand{\headrulewidth}{0pt}
	\renewcommand{\footrulewidth}{0.5pt}
	\fancyfoot[LE,RO]{\thepage}
	\fancyfoot[LO,RE]{Qualitätssicherung Wintersemester 2012/2013}
}

% no indented paragraphs
\usepackage{parskip}

\setkomafont{disposition}{\normalfont\bfseries}

\usepackage{natbib}

\usepackage{amsfonts}

\usepackage{verbatim}

\begin{document}

\titlehead{%
	\includegraphics[width=0.9\linewidth]{img/hska_logo.png}
}

\title{Qualitätssicherung}
\author{%
	Daniel Bälz, Mat.-Nr.: 31074 \\
	Fabian Freimüller, Mat.-Nr.: 30684 \\
}
\date{Wintersemester 2012/2013}
\publishers{%
    \textbf{Betreuer:} Prof.\,Dr. Dirk W. Hoffmann
}
\maketitle

\tableofcontents

\clearpage

\section{Einleitung}
\label{sec:intro}
Ziel der Blockveranstaltung Qualitätssicherung ist es, in die Thematik der Software-Qualitätssicherung einzuführen.
Kombiniert wird hierbei die Vorlesung mit einer praktischen Aufgabe, die einen Teilbereich des Vorlesungsinhalt umsetzt. Im Rahmen der Vorlesung wurde erläutert wie aus zwei gegebenen Sequenzen die Längste-Gemeinsame-Teilfolge (LGT) mit einer maximalen Komplexität von $O(n^2)$ ermittelt wird.
Mit Hilfe dieser Teilfolge ist es möglich eine Verschmelzung von Dateien durchzuführen und eine Teilmenge der entstehenden Konflikte automatisiert zu lösen.
Aufbauend auf die in der Vorlesung gelehrten Konzepte bestand die erste Aufgabe daraus, ein Programm zur Verschmelzung von zwei unabhängig weiterentwickelten Revision V1 und V2 aus der Ausgangsdatei V selbstständig ohne Zuhilfenahme bestehender Verschmelzungs-Algorithmen zu implementieren. Auftretende Konflikte sollten wenn möglich automatisiert gelöst werden. Im Fall von nicht lösbaren Konflikten soll der Anwender über den Konflikt informiert werden, so dass er eine manuelle Verschmelzung vornehmen kann.
Die zweite Aufgabe bestand daraus, dass in der ersten Aufgabe erstellte Programm mittels geeigneter Testfälle und eines Programms zur Überprüfung der Zeilenabdeckung zu testen.


Der Quellcode des Programm, die Testfälle und diese Dokumentationen finden sich unter \url{https://github.com/Ari42/filemerger}.

\section{Wahl der Programmiersprache}
\label{sec:proglang}
Für die Aufgaben war die Implementierungssprache freigestellt, so fern beide Aufgaben unter den oben genannten Bedingungen mit der Sprache realisierbar waren.
Aufgrund der Tatsache, dass beide Gruppenmitglieder in der Programmierung mit Java die größte Erfahrung haben und mit \citep{www:ECLEMMA} eine bereits verwendetes Tool zum Test der Zeilenabdeckung vorhanden ist wurde diese Programmiersprache gewählt.

\section{Design der Benutzerschnittstellen}
\label{sec:design}
Um das Programm möglichst flexibel verwenden zu können wurde es als Kommandozeilenprogramm gestaltet.
Das Programm übernimmt als Parameter die drei benötigten Dateien, mit deren Hilfe die Drei-Wege-Verschmelzung durchgeführt wird. Das Programm kann in der Kommandozeile wie folgt aufgerufen werden:

\lstset{language=sh, numbers=none, xleftmargin=0pt}
\begin{lstlisting}
java -jar FileMerger V.txt V1.txt V2.txt
\end{lstlisting}
Wie dem Dateinamen zu entnehmen ist handelt es sich beim ersten Parameter um die Ausgangsdatei $V$,
die beiden folgenden Parameter stellen die zu verschmelzenden Revisionen $V1$ und $V2$ dar.
Dies widerspricht der bei diff3 gängigen Reihenfolge $V1$, $V$, $V2$, jedoch halten die Gruppenmitglieder das gewählte Format für verständlicher.

Das Ergebnis der Verschmelzung wird vom Programm auf der Standardausgabe ausgegeben. Konflikte werden hierbei unter Angabe der konfliktbehafteten Stelle in folgender Form angezeigt:

\begin{verbatim}
Merge Error: File2: ABC and File3: DEF
\end{verbatim}


Die Ausgabe wird nicht in eine Datei geschrieben. Bei gewünschter Speicherung in eine Datei wird der Benutzer angehalten, die Ausgabe selbst mittels Ausgabeumleitung in eine Datei zu schreiben.

\section{Algorithmus}
\label{sec:algorithm}
Bevor die Verschmelzung der Dateien durchgeführt werden kann, muss auf Zeilenebene die längste gemeinsame Teilfolge aller drei Eingabedateien ermittelt werden.
Hierzu wird der unter \citet{www:EP96} beschriebenen Algorithmus verwendet. Dieser Algorithmus nutzt die aufsteigende dynamische Programmierung,
welche eine deutlich kürzere Laufzeit im Vergleich zu einer rekursiven Ermittlung des LGT ermöglicht.

Der erwähnte Algorithmus zur Ermittlung des LGT bekommt setzt mit Hilfe der Eingabedateien die folgenden Schritte um:

\begin{itemize}
\item Erstellung einer genullten Matrix aus den Eingabedateien zwecks späterer Speicherung der Vergleichsresultate.
\item Aufsteigende Iteration durch die genullte Matrix. Dieser Durchlauf beginnt rechts unten und endet links oben.
\item Bei einer Übereinstimmung zweier Listenelemente wird der Wert, abhängig vom diagonal unterliegenden Feld, um 1 erhöht.
\item Ist keine weitere Übereinstimmung vorhanden erhalten die anderen Felder den Wert des rechts bzw. unten angrenzenden Feldes.
\end{itemize}

Durch dieses Vorgehen ist am Ende des Durchlaufes im ersten Feld der Matrix die Anzahl ablesbar.
Um nun die längste gemeinsame Teilfolge zu ermitteln muss erneut über die gewonnene Matrix und die einzelnen Listenelemente der Eingabeparameter iteriert werden.
Hierfür werden die diagonalen Übergänge in der Matrix gesucht, da dort die Listenelemente übereinstimmen. Diese werden dann zur Ergebnismenge hinzugefügt.
Der beschriebene Algorithmus ermittelt den LGT innerhalb von $O(nm)$ , bei unterschiedlich langen Dateien, und in $O(n^2)$ bei gleich langen Dateien. Somit erfüllt der Algorithmus die gewünschte Laufzeitforderung.


\section{Implementierung}
\label{sec:implementation}
Für die Implementierung des Algorithmus wurden in Java ein Programm mit zwei Klassen erstellt, welche die Funktionalität beinhalten.

\subsection{Klasse Main}
\label{sec:classmain}
Die Klasse \texttt{Main} enthält als statische Methode die \texttt{main}-Methode, welche die Eingabedateien als Parameter übergeben bekommt.
Dieser Methode obliegt es, die Parameter auf ihre Korrektheit zu prüfen und anschließend die in \texttt{File}-Objekte gewandelten Dateien
an die die Verschmelzung durchführende \texttt{Merge}-Klasse weiter zu geben.
Das Verschmelzungsergebnis und der Warnhinweis im Konfliktfall wird durch diese Klasse ausgegeben.

\subsection{Klasse Merge}
\label{sec:classmerge}
Die Erstellung des LGT und die darauf aufbauende Verschmelzung der Dateien $V1$ und $V2$ ist Aufgabe der Klasse \texttt{Merge}. Der Konstruktor nimmt aus der oben genannten \texttt{Main}-Klasse die \texttt{File}-Objekte entgegen. Danach kann mit der \texttt{mergeFiles}-Methode die Verschmelzung durchgeführt werden, wobei diese Methode mittels einen Boolean anzeigt ob die Verschmelzung erfolgreich war. Die Ergebnisliste kann anschließend über die \texttt{getMergedList}-Methode abgerufen werden. Dies ermöglicht es, die Überprüfung ob die Verschmelzung möglich ist von der Rückgabe der Ergebnisliste zu trennen.

\subsubsection{mergeFiles()}
\label{sec:mergefiles}
Innerhalb dieser Datei wird aus den Eingabedateien durch die \texttt{makeHash}-Methode Hashlisten erstellt. Aus diesen wiederum wird mit der \texttt{lcs}-Methode drei Listen erzeugt.
Bei diesen Listen handelt es sich um die längste gemeinsame Teilfolge der Kombinationen aus $V$ und $V1$, $V$ und $V2$ und aus den beiden zuvor erstellten Listen. Mit Hilfe dieser Listen verschmelzt die \texttt{merge}-Methode die Dateien $V1$ und $V2$. Falls die Verschmelzung ohne Fehler möglich war gibt die Methode \texttt{true} zurück, im Fehlerfall \texttt{false}.

\subsubsection{makeHash(File file)}
\label{sec:makehash}
Der Inhalt der übergebenen Datei wird durch die \texttt{makeHash}-Methode zeilenweise eingelesen.
Die eingelesenen Zeilen werden dann als SHA-1 Hash in eine \texttt{ArrayList} abgespeichert. Da es sich hierbei um eine sortierte Liste handelt kann im weiteren Verlauf mittels Indexzugriff auf die Hashwerte jeder Zeile zugegriffen werden.

\subsubsection{lcs(List<String> list1, List<String> list2)}
\label{sec:lcs}
Die übergebenen Listen benutzt die \texttt{lcs}-Methode um den oben beschriebenen Algorithmus zur Ermittlung der längsten gemeinsamen Teilfolge zu implementieren.
Hierzu werden immer zwei Listen als Eingabeparameter übergeben, aus diesen die längste gemeinsame Teilfolge ermittelt und diese als ArrayList zurück gegeben.

\subsubsection{merge()}
\label{sec:merge}
Mit der \texttt{merge}-Methode wird die eigentliche Verschmelzung, unter Zuhilfenahme der LGT-Listen, erstellt.

Für die Beschreibung der Methode sollen folgende Definitionen gelten:
\begin{itemize}
\item $V1 , V2$ und $V$ beschreiben die beiden geänderten Dateien/Listen und die Ausgangsdatei/Liste.
\item $L_{V_{1}V}$ sei die Längste-Gemeinsame-Teilfolge von $V1$ und $V$
\item $L_{V_{2}V}$ sei die Längste-Gemeinsame-Teilfolge von $V2$ und $V$
\item $L_{Ges}$ sei die Längste-Gemeinsame-Teilfolge von $L_{V_{1}V}$ und $L_{V_{2}V}$
\end{itemize}

Die \texttt{merge}-Methode benutzt nun die oben genannten LGT-Listen und versucht mit diesen die Dateien/Listen zu verschmelzen,
indem die Methode über die einzelnen Elemente der geänderten Dateien/Listen iteriert. Diese Verschmelzung wir in einer Ergebnisliste gespeichert.
Zur Ermittlung des Ergebnisses wird folgendes solange durchgeführt bis kein Element in einer der Listen verbleibt:
\begin{itemize}
\item Wenn $V1(i) = V2(j)$ gilt, füge die Zeile dem Ergebnis hinzu
\item Ist $V1(i) = L_{Ges}(m)$ , füge die Zeile $V2(j)$ dem Ergebnis hinzu, da diese neu eingefügt wurde.
\item Ist $V2(j) = L_{Ges}(m)$, füge die Zeile $V1(i)$ dem Ergebnis hinzu.
\item Wenn $V1(i) = L_{V_{1}V}(k)$ gilt, füge $V2(j)$ dem Ergebnis hinzu, da $V1$ die Zeile beibehalten hat und in $V2$ eine Änderung stattfand.
\item Wenn $V2(j) = L_{V_{2}V}(l)$ gilt, füge $V1(i)$ dem Ergebnis hinzu, da $V2$ die Zeile beibehalten hat und in $V1$ eine Änderung stattfand.
\item Der letzte Fall beschreibt einen Konflikt bei der Verschmelzung, da sowohl $V1(i)$ als auch $V2(j)$ neu eingefügt worden sind. In diesem Fall werden beide Zeilen mit der in Abschnitt~\ref{sec:design} beschriebenen Fehlermeldung der Ergebnisliste hinzugefügt.
\end{itemize}
Enthält eine Liste keine Elemente mehr, so werden die Elemente der längeren Liste dem Ergebnis hinzugefügt.


\section{Test}
\label{sec:test}

Wie in Abschnitt~\ref{sec:intro} beschrieben sollte als zweite Aufgabe mittels geeigneter Testfälle die Zeilenabdeckung getestet werden.
Hierbei sollte eine Zeilenabdeckung von 100\%, sowohl für den Quelltext des Programm als auch bei den Testfällen erreicht werden.
Gefordert war neben der Zeilenabdeckung des bestehenden Programm auch ein Testfall, bei dem nicht genügend Arbeitsspeicher beim Einlesen der Datei zu Verfügung steht.

Um diese Anforderungen umzusetzen wurden Unit-Tests mit Hilfe von \citep{www:JUNIT} erstellt. Die Zeilenabdeckung der Unit-Tests wurde mit dem \citep{www:ECLEMMA} überprüft, welches sich ebenso wie JUnit in die von den Gruppenmitgliedern genutzte Entwicklungsumgebung Eclipse eingliedern lässt.

\subsection{Klasse FileErros}
\label{sec:fileerrors}
In der Klasse \texttt{FileErrors} wurden Fehlerfälle abgedeckt, die mit den Parametern bei Programmstart zusammen hängen.
Hierbei wurde die korrekte Anzahl der Parameter getestet, da die Verschmelzung nur bei genau drei Dateien möglich ist.
Dazu wurden jeweils den Testfällen zu wenige, zu viel als auch nicht vorhandene Dateinamen übergeben.

\subsection{Klasse TestFileUnreadable}
\label{sec:fileunreadable}
Ziel der Klasse \texttt{TestFileUnreadable} war es, das Verhalten des Programm bei nicht lesbaren Eingabedateien zu testen.
Hierzu wurde eine Datei auf Dateisystemebene als nicht lesbar gekennzeichnet und anschließend im Testfall übergeben.
Um diesen Testfall auch für Dritte, welche den Quelltext aus dem oben genannten Repository herunterladen, durchführbar zu machen, wurde in der dem Projekt beiliegenden \texttt{README.md} ein Hinweis eingefügt.

\subsection{Klasse FilesEqual}
\label{sec:filesequal}
Sind die zu verschmelzenden Dateien identisch, so ist die Erstellung des LGT und das Verschmelzen nicht notwendig. Statt dessen wird die erste Datei als Verschmelzungsergebnis ausgegeben, was die Laufzeit des Programm erheblich verkürzt. Um diesen Fall zu testen wurde in der \texttt{FilesEqual}-Klasse identische, zu verschmelzende Dateien.

\subsection{Klasse GivenFiles}
\label{sec:givenfiles}
Die vom Betreuer ausgegeben Dateien zum Test des Programmverhalten wurden in der Klasse \texttt{GivenFiles} getestet. Mit diesen Dateien konnten aufgrund ihres Aufbaus einen Großteul der Zeilenabdeckung abgedeckt werden. Das aus dieser Verschmelzung erzeugte Verschmelzungsergebnis diente zudem bei der Abgabe der ersten Aufgabe zur Überprüfung der Korrektheit des Algorithmus.

\subsection{Klasse FilesLonger}
\label{sec:fileslonger}
Ein Spezialfall deckt die Klasse \texttt{FilesLonger} ab. Enthält eine der beiden Dateien keine Elemente mehr, so werden die Elemente der längeren Datei dem Ergebnis hinzugefügt.
Um dieses Verhalten abzudecken wurden für die beiden Fälle, dass die erste bzw. zweite Datei länger ist, Unit-Tests geschrieben.

\subsection{Klasse NotEnoughMemory}
\label{sec:notenoughmemory}
Wie oben beschrieben sollte auch abgedeckt werden, dass nicht genügend Arbeitsspeicher zum Einlesen der Dateien zur Verfügung steht.
Da dieser Fall bei einer standardgemäß konfigurierten Java Virtual Machine nicht auftritt mussten an dieser Anpassungen vorgenommen werden.
Als erster Schritt wurde eine weitere Startkonfiguration für die Zeilenabdeckung erzeugt. Dieser Konfiguration wurde als VM-Parameter \texttt{-Xmx5M} übergeben.
Dieser Parameter begrenzt den verfügbaren Heapspeicher der JVM auf maximal 5 MB. In Verbindung mit einer 5 MB großen Testdatei konnte so das gewünschte Verhalten provoziert werden.
Im Anschluss wurde ein Testfall geschrieben welcher die Datei innerhalb einer \texttt{try/catch}\-Struktur aufruft und entsprechend eine \texttt{OutOfMemoryError}-Exception wirft.

Da nun zwei Startkonfigurationen vorhanden waren, welche gemeinsam das gesamte Verhalten abdecken, mussten diese kombiniert werden.
Die Gruppenmitglieder nutzten hierfür die Möglichkeit innerhalb des \citep{www:ECLEMMA} zwei Testdurchläufe zu verknüpfen.
Dadurch ist es möglich, sowohl mit ausreichendem als auch mit zu geringem Arbeitsspeicher zu testen und anschließend durch Kombination 100\% Testabdeckung zu erhalten.
Um diesen Testfall auch für Dritte, welche den Quelltext aus dem oben genannten Repository herunterladen, durchführbar zu machen, wurde in der dem Projekt beiliegenden \texttt{README.md} ein Hinweis eingefügt.

\clearpage
\appendix

\bibliographystyle{plainnat}
\bibliography{BIB}

\end{document}